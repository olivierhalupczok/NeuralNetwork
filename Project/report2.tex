% !TeX spellcheck = en_GB
%%%%%%%%%%%%%%%%%%%%%%%%%%%%%%%%%%%%%%%%%%%%%%
%                                            %
%  R E P O R T   T E M P L A T E             %
%                                            %
%%%%%%%%%%%%%%%%%%%%%%%%%%%%%%%%%%%%%%%%%%%%%%


\documentclass[12pt,a4paper,twoside]{article}

\usepackage{amsmath,amssymb}
\usepackage[utf8]{inputenc}                                      
%\usepackage[OT4]{fontenc}      
\usepackage[T1]{fontenc}                            
\usepackage[british]{babel}                           
\usepackage{indentfirst} 
\usepackage[dvips]{graphicx}
\usepackage{tabularx}
\usepackage{color}
\usepackage{hyperref} 
\usepackage{fancyhdr}
\usepackage{listings}
\usepackage{booktabs}
\usepackage{ifpdf}
\usepackage{mathtext} 
%\makeindex  
\usepackage{lmodern}
%\usepackage[osf]{libertine}
\usepackage{filecontents}
\usepackage{ifthen}


\usepackage{tikz}


\newcounter{nextYear}
\setcounter{nextYear}{\the\year}
\stepcounter{nextYear}

 

%%%% ZYWA PAGINA %%%%%%%%%%%
\newcommand{\tl}[1]{\textbf{#1}} 
\pagestyle{fancy}
\renewcommand{\sectionmark}[1]{\markright{\thesection\ #1}}
\fancyhf{} % usuwanie bieżących ustawień
\fancyhead[LE,RO]{\small\bfseries\thepage}
\fancyhead[LO]{\small\bfseries\rightmark}
\fancyhead[RE]{\small\bfseries\leftmark}
\renewcommand{\headrulewidth}{0.5pt}
\renewcommand{\footrulewidth}{0pt}
\addtolength{\headheight}{0.5pt} % pionowy odstęp na kreskę
\fancypagestyle{plain}{%
\fancyhead{} % usuń p. górne na stronach pozbawionych numeracji
\renewcommand{\headrulewidth}{0pt} % pozioma kreska
}

%%%%%   LISTINGI %%%%%%%%
% ustawienia listingu programow

\lstset{%
language=C++,%
commentstyle=\textit,%
identifierstyle=\textsf,%
keywordstyle=\sffamily\bfseries, %
%captionpos=b,%
tabsize=3,%
frame=lines,%
numbers=left,%
numberstyle=\tiny,%
numbersep=5pt,%
breaklines=true,%
morekeywords={node,string,ref,params_result},%
escapeinside={(*@}{@*)},%
%basicstyle=\footnotesize,%
%keywords={double,int,for,if,return,vector,matrix,void,public,class,string,%
%float,sizeof,char,FILE,while,do,const}
}
%%%%%%%%%%%%%%%%%%%%%%%%%%%%%%%%%%%%%%%%%%%%%%%%%%%%%%%%%%%%%%%%%%%%%%%

%%%%%%%%%  NOTKI NA MARGINESIE %%%%%%%%%%%%%
% mala zmiana sposobu wyswietlania notek bocznych
\let\oldmarginpar\marginpar
\renewcommand\marginpar[1]{%
  {\linespread{0.85}\normalfont\scriptsize%
\oldmarginpar[\hspace{1cm}\begin{minipage}{3cm}\raggedleft\scriptsize\color{black}\textsf{#1}\end{minipage}]%    left pages
{\hspace{0cm}\begin{minipage}{3cm}\raggedright\scriptsize\color{black}\textsf{#1}\end{minipage}}% right pages
}%
}
% % % % % % % % % % % % % % % % % % % % % % % % % % % % % % % %

%%%% WYSWIETLANIE AKTUALNEGO ROKU AKADEMICKIEGO %%%%%%%%%%%
\newcounter{rok}
\newcommand{\rokakademicki}{%
   \setcounter{rok}{\number\year}%
   \ifthenelse{\number\month<10}%
   {\addtocounter{rok}{-1}}% rok akademicki zaczal sie w pazdzierniku poprzedniego roku
   {}%                       rok akademicki zaczyna sie w pazdzierniku tego roku
   \arabic{rok}/\addtocounter{rok}{1}\arabic{rok}
}
%%%%%%%%%%%%%%%%%%%%%%%%%%%%%%%%%%%%%%%


%%%% LISTA UWAG %%%%%%%%%
\usepackage{color}
\definecolor{brickred}      {cmyk}{0   , 0.89, 0.94, 0.28}

\makeatletter \newcommand \kslistofremarks{\section*{Remarks} \@starttoc{rks}}
\newcommand\l@uwagas[2]
{\par\noindent \textbf{#2:} %\parbox{10cm}
   {#1}\par} \makeatother


\newcommand{\ksremark}[1]{%
   {{\color{brickred}{[#1]}}}%
   \addcontentsline{rks}{uwagas}{\protect{#1}}%
}


%%%%%%%%%%%%%%%%%%%%%%%%%
%%%%%%%%%%%%%%%%%%%%%%%%%
%%%%%%%%%%%%%%%%%%%%%%%%%
%%%%%%%%%%%%%%%%%%%%%%%%%
%%%%%%%%%%%%%%%%%%%%%%%%%
%%%%%%%%%%%%%%%%%%%%%%%%%
%%%%%%%%%%%%%%%%%%%%%%%%%
%%%%%%%%%%%%%%%%%%%%%%%%%
%%%%%%%%%%%%%%%%%%%%%%%%%
%%%%%%%%%%%%%%%%%%%%%%%%%
%%%%%%%%%%%%%%%%%%%%%%%%%
%%%%%%%%%%%%%%%%%%%%%%%%%



% autor:
\fancyhead[RE]{\small\bfseries Olivier Halupczok}  



%%%%%%%%%%% REPORT %%%%%%%%%%%

\begin{document}
\frenchspacing
\thispagestyle{empty}
\begin{center}
{\Large\sf Politechnika Śląska   % Alma Mater

Wydział Informatyki, Elektroniki i Informatyki

}

\vfill

 

\vfill\vfill

{\Huge\sffamily\bfseries Fundamentals of Computer Programming\par}

\vfill\vfill

{\LARGE\sf «Knapsack problem»} 


\vfill \vfill\vfill\vfill

%%%%%%%%%%%%%%%%%%%%%%%%%%%%





\begin{tabular}{ll}
	\toprule
	author     & Olivier Halupczok                           \\
	instructor & dr inż. Sławomir Lasota \\
	year       & \2021-2022                     \\
	lab group  & Friday, 14:30 -- 16:00             \\
	deadline   & 2022-02-20                         \\
	\bottomrule
	           &
\end{tabular}

\end{center}
%%% koniec strony  tytulowej

%%%%%%%%%%%%%%%%%%%%%%%%%%%%%%%%%%%%%%%%%%%%%%%%%%%%%%%%%%%%%%%%%%%%%%%%%
\cleardoublepage
%%%%%%%%%%%%%%%%%%%%%%%%%%%%%%%%%%%%%%%%%%%%%%%%%%%%%%%%%%%%%%%%%%%%%%%%%

%%%%%%%%%%%%%%%%%%%%%%%%%%%%%%%%%%%%%%%%%%%%%%%%%%%%%%%%%%%%%%%%%%%%%%%%%
\section{Project's topic}
% \marginpar{Put precise description of your task – copy it from your instructor's task list.}
Implement a program that solves the knapsack problem with a genetic algorithm. An input file holds a set of
items with their masses and values, e.g.:%
\begin{verbatim}
  tvset 10.0 3300
  mug 0.5 20
  laptop 2.2 4000
  screen 4.5 1500
  tablet 0.5 900
  washer 35.0 2200
camera 1.2 3500
teapot 1.5 130
\end{verbatim}
The aim of the project is to find a subset of items with maximal value and total weight that does not exceed the
capacity of the knapsack. Exempli gratia: for a knapsack with capacity 2.5 kg the best solution is:
\begin{verbatim}
  tablet 0.5 900
  camera 1.2 3500
\end{verbatim}
The necessary condition for this project is an implementation of a genetic algorithm with selection and
crossover operators (we leave out a mutation operator). An output file holds the best solutions found in all
generations, e.g.:
generation 1, weight 2.0, value 150:
\begin{verbatim}
  mug 0.5 20
  teapot 1.5 130
  generation 2, weight 1.7, value 3520:
  mug 0.5 20
  camera 1.2 3500
\end{verbatim}
% \marginpar{Use a serif font for the main text. Justify the text (left and right alignment).}
The program is called with parameters: 
\begin{tabular}{ll}
\texttt{-i}  & input file name\\
\texttt{-o}  & output file with the best solutions in all generations\\
\texttt{-c}  & knapsack capacity\\
\texttt{-g}  & number of generations\\
\texttt{-n}  & number of individuals in a generation\\
\end{tabular}

%%%%%%%%%%%%%%%%%%%%%%%%%%%%%%%%%%%%%%%%%%%%%%%%%%%%%%%%%%%%%%%%%%%%%%%%%
\section{Analysis of the task}
% \marginpar{Analyse the task before you start to implement it! Describe data structures and algorithms.}

The task focuses on selection of the best subset of items that meets the conditions specified by the given parameters.

\subsection{Data structures}
% \marginpar{Describe data structures you use in your project (lists, trees, heaps, , \ldots). Explain your choice. Add a figure.}
The data structures that the program uses are array, vector, map. Arrays and vectors are used interchangeably depending on the convenience of the structure's implementation and its intention. 
The target of using arrays was to pass the arguments' array obtained in the runtime of the program's execution to the functions responsible for displaying help in the command line.
The vectors have been vastly used in the source code to take advantage of the effortless extension of the structure's size and to use methods declared in the class definition to keep code sustainable and readable.
The map structure has been used to simplify the recognition and handling of unknown flags. The procedure of checking flags starts with the validation of flags' quantity. If execution parameters meet the condition, the program proceeds to examine the flags' identifiers and their values. Meanwhile, the array of arguments converts into a map, where the key is the identifier of the flag, and the value is passed as the value assigned to the proper key. The program tries to read every expected flag. If it endeavours to get an unexisting key, the map's size increments. Comparing map's size delta simply detect if the unknown argument has been passed.

\subsection{Algorithms}
The program add items to the vector of all possible items. In another step of program's execution the Generations' factory generates first generation and makes random individuals. All individuals are represented as objects of Individual class and have binary mask to determine state of possession specified objects. During process of generating every generation, the program sort individuals by their summary value(descending) and weight(ascending, if there are at least two individuals with the same summary value). Then it takes only the best half to another generation, and populate other half of the individuals vector by using crossover process to remain the quantity of individuals in each generation equal. The algorithm execution time depends on items, generations, individuals number.

%%%%%%%%%%%%%%%%%%%%%%%%%%%%%%%%%%%%%%%%%%%%%%%%%%%%%%%%%%%%%%%%%%%%%%%%%
\section{External specification}
\label{sec:external}
This is a command line program.%
The program requires names of input and output files, it also needs numbers of generations, individuals in each generation and knapsack's capacity. Put input file name after  \texttt{-i} flag, output file name after \texttt{-o} flag, number of generations after \texttt{-g} flag, number of individuals in each generation after \texttt{-n} flag, knapsack's capacity after \texttt{-c} flag eg:
\begin{verbatim}
program -i input-file -o output-file -g generations-num -n individuals-in-generations-num -c knapsack-capacity-num
program -o output-file -i input-file -g generations-num -n individuals-in-generations-num -c knapsack-capacity-num
\end{verbatim}
Both files are text files. The flags might be provided in any order. The program called with unknown flags or with no flags print help.

Program call
\begin{verbatim}
program 
program -h
\end{verbatim}
prints a short manual. Program called with incorrect parameters prints an error message and prints help.

Incorrect file names are detected and cause a message:
\begin{verbatim}
input file does not exist or cannot be open!
\end{verbatim}


%%%%%%%%%%%%%%%%%%%%%%%%%%%%%%%%%%%%%%%%%%%%%%%%%%%%%%%%%%%%%%%%%%%%%%%%%
\section{Internal specification}\label{sec:internal}
The program is implemented with object-oriented, structural and functional paradigm. User interface is separated from program's logic.

 

\subsection{Program overview}
The \lstinline|main| function checks number of arguments passed to the program and creates a new Options object, which constructor check parameters of the program. If the verification is negative, an appropriate message is printed. In case of positive verification, data are read and saved into vector, which is passed, simultaneously with the options object, to the constructor of an \lstinline|GenerationsFactory| object. Then \lstinline|GenerationsFactory::makeGenerations| method performs genetic algorithm by creating first generation using objects of classes: \lstinline|Generation|, \lstinline|Individual|, and then it creates every another generation of individuals. Finally the program prints data into an output file.

\subsection{Description of types and functions}
Description of types and functions is moved to the appendix.

 

\section{Testing}
% \marginpar{Describe how you have tested your program.}
The program has been tested with various types of files. Incorrect files (with no numbers, numbers in incorrect format, strings with some invalid whitespaces, \ldots) are detected and an error message is printed. An empty input file does not cause failure – an empty output file is created. Maximal number value(\lstinline!double!) in an input file is approximately 1.8e+308. Maximal input file size handled by the program is \mbox{1.57$\,$GB}. Larger files result in a bad allocation error. The program has no memory leaks.
 


\section{Conclusions}

The program implements a solution to a knapsack problem using genetic algorithm. The most challenging task is creating a crossover system where a lot of data can be processed iteratively.

For some parameters the program elaborates incorrect results on some machines. This is caused by specification of the algorithm which does not includes an evolution coeeficient. Algorithm also base on a small dose of randomness, and it can be succesfull if generation/individuals number is insufficient.


\bibliographystyle{plain}
\bibliography{\jobname}

 
 
\cleardoublepage

\rule{0cm}{0cm}

\vfill

\begin{center}
\Huge\bfseries Appendix\\Description of types and functions\par
\end{center}

\vfill 

\rule{0cm}{0cm}

\end{document}
% Koniec wieńczy dzieło.

% !TeX spellcheck = en_GB
%%%%%%%%%%%%%%%%%%%%%%%%%%%%%%%%%%%%%%%%%%%%%%
%                                            %
%  R E P O R T   T E M P L A T E             %
%                                            %
%%%%%%%%%%%%%%%%%%%%%%%%%%%%%%%%%%%%%%%%%%%%%%


\documentclass[12pt,a4paper,twoside]{article}

\usepackage{amsmath,amssymb}
\usepackage[utf8]{inputenc}                                      
%\usepackage[OT4]{fontenc}      
\usepackage[T1]{fontenc}                            
\usepackage[british]{babel}                           
\usepackage{indentfirst} 
\usepackage[dvips]{graphicx}
\usepackage{tabularx}
\usepackage{color}
\usepackage{hyperref} 
\usepackage{fancyhdr}
\usepackage{listings}
\usepackage{booktabs}
\usepackage{ifpdf}
\usepackage{mathtext} 
%\makeindex  
\usepackage{lmodern}
%\usepackage[osf]{libertine}
\usepackage{filecontents}
\usepackage{ifthen}


\usepackage{tikz}


\newcounter{nextYear}
\setcounter{nextYear}{\the\year}
\stepcounter{nextYear}

 

%%%% ZYWA PAGINA %%%%%%%%%%%
\newcommand{\tl}[1]{\textbf{#1}} 
\pagestyle{fancy}
\renewcommand{\sectionmark}[1]{\markright{\thesection\ #1}}
\fancyhf{} % usuwanie bieżących ustawień
\fancyhead[LE,RO]{\small\bfseries\thepage}
\fancyhead[LO]{\small\bfseries\rightmark}
\fancyhead[RE]{\small\bfseries\leftmark}
\renewcommand{\headrulewidth}{0.5pt}
\renewcommand{\footrulewidth}{0pt}
\addtolength{\headheight}{0.5pt} % pionowy odstęp na kreskę
\fancypagestyle{plain}{%
\fancyhead{} % usuń p. górne na stronach pozbawionych numeracji
\renewcommand{\headrulewidth}{0pt} % pozioma kreska
}

%%%%%   LISTINGI %%%%%%%%
% ustawienia listingu programow

\lstset{%
language=C++,%
commentstyle=\textit,%
identifierstyle=\textsf,%
keywordstyle=\sffamily\bfseries, %
%captionpos=b,%
tabsize=3,%
frame=lines,%
numbers=left,%
numberstyle=\tiny,%
numbersep=5pt,%
breaklines=true,%
morekeywords={node,string,ref,params_result},%
escapeinside={(*@}{@*)},%
%basicstyle=\footnotesize,%
%keywords={double,int,for,if,return,vector,matrix,void,public,class,string,%
%float,sizeof,char,FILE,while,do,const}
}
%%%%%%%%%%%%%%%%%%%%%%%%%%%%%%%%%%%%%%%%%%%%%%%%%%%%%%%%%%%%%%%%%%%%%%%

%%%%%%%%%  NOTKI NA MARGINESIE %%%%%%%%%%%%%
% mala zmiana sposobu wyswietlania notek bocznych
\let\oldmarginpar\marginpar
\renewcommand\marginpar[1]{%
  {\linespread{0.85}\normalfont\scriptsize%
\oldmarginpar[\hspace{1cm}\begin{minipage}{3cm}\raggedleft\scriptsize\color{black}\textsf{#1}\end{minipage}]%    left pages
{\hspace{0cm}\begin{minipage}{3cm}\raggedright\scriptsize\color{black}\textsf{#1}\end{minipage}}% right pages
}%
}
% % % % % % % % % % % % % % % % % % % % % % % % % % % % % % % %

%%%% WYSWIETLANIE AKTUALNEGO ROKU AKADEMICKIEGO %%%%%%%%%%%
\newcounter{rok}
\newcommand{\rokakademicki}{%
   \setcounter{rok}{\number\year}%
   \ifthenelse{\number\month<10}%
   {\addtocounter{rok}{-1}}% rok akademicki zaczal sie w pazdzierniku poprzedniego roku
   {}%                       rok akademicki zaczyna sie w pazdzierniku tego roku
   \arabic{rok}/\addtocounter{rok}{1}\arabic{rok}
}
%%%%%%%%%%%%%%%%%%%%%%%%%%%%%%%%%%%%%%%


%%%% LISTA UWAG %%%%%%%%%
\usepackage{color}
\definecolor{brickred}      {cmyk}{0   , 0.89, 0.94, 0.28}

\makeatletter \newcommand \kslistofremarks{\section*{Remarks} \@starttoc{rks}}
\newcommand\l@uwagas[2]
{\par\noindent \textbf{#2:} %\parbox{10cm}
   {#1}\par} \makeatother


\newcommand{\ksremark}[1]{%
   {{\color{brickred}{[#1]}}}%
   \addcontentsline{rks}{uwagas}{\protect{#1}}%
}


%%%%%%%%%%%%%%%%%%%%%%%%%
%%%%%%%%%%%%%%%%%%%%%%%%%
%%%%%%%%%%%%%%%%%%%%%%%%%
%%%%%%%%%%%%%%%%%%%%%%%%%
%%%%%%%%%%%%%%%%%%%%%%%%%
%%%%%%%%%%%%%%%%%%%%%%%%%
%%%%%%%%%%%%%%%%%%%%%%%%%
%%%%%%%%%%%%%%%%%%%%%%%%%
%%%%%%%%%%%%%%%%%%%%%%%%%
%%%%%%%%%%%%%%%%%%%%%%%%%
%%%%%%%%%%%%%%%%%%%%%%%%%
%%%%%%%%%%%%%%%%%%%%%%%%%



% autor:
\fancyhead[RE]{\small\bfseries Olivier Halupczok}  



%%%%%%%%%%% REPORT %%%%%%%%%%%

\begin{document}
\frenchspacing
\thispagestyle{empty}
\begin{center}
{\Large\sf Politechnika Śląska   % Alma Mater

Wydział Informatyki, Elektroniki i Informatyki

}

\vfill

 

\vfill\vfill

{\Huge\sffamily\bfseries Computer Programming\par}

\vfill\vfill

{\LARGE\sf «Neural Network»} 


\vfill \vfill\vfill\vfill

%%%%%%%%%%%%%%%%%%%%%%%%%%%%





\begin{tabular}{ll}
	\toprule
	author     & Olivier Halupczok                           \\
	instructor & dr inż. Piotr Fabian \\
	year       & \2021-2022                     \\
	lab group  & even Tuesday, 10:15 -- 11:45             \\
	deadline   & 2022-06-24                         \\
	\bottomrule
	           &
\end{tabular}

\end{center}
%%% koniec strony  tytulowej

%%%%%%%%%%%%%%%%%%%%%%%%%%%%%%%%%%%%%%%%%%%%%%%%%%%%%%%%%%%%%%%%%%%%%%%%%
\cleardoublepage
%%%%%%%%%%%%%%%%%%%%%%%%%%%%%%%%%%%%%%%%%%%%%%%%%%%%%%%%%%%%%%%%%%%%%%%%%

%%%%%%%%%%%%%%%%%%%%%%%%%%%%%%%%%%%%%%%%%%%%%%%%%%%%%%%%%%%%%%%%%%%%%%%%%
\section{Project's topic}
% \marginpar{Put precise description of your task – copy it from your instructor's task list.}
Simple Neural Network defined as a class that supports Neurons., configuring a user-defined connection structure, selected learning methid.
\end{verbatim}
% \marginpar{Use a serif font for the main text. Justify the text (left and right alignment).}

% The program is called with parameters: 
% \begin{tabular}{ll}
% \texttt{-i}  & input file name\\
% \texttt{-o}  & output file with the best solutions in all generations\\
% \texttt{-c}  & knapsack capacity\\
% \texttt{-g}  & number of generations\\
% \texttt{-n}  & number of individuals in a generation\\
% \end{tabular}

%%%%%%%%%%%%%%%%%%%%%%%%%%%%%%%%%%%%%%%%%%%%%%%%%%%%%%%%%%%%%%%%%%%%%%%%%
\section{Analysis of the task}
% \marginpar{Analyse the task before you start to implement it! Describe data structures and algorithms.}

The task focuses on the analyzing data given as an input dataset and predicting the feature of the object. It requires implementation of the learning algorithm and minimizing loss of the entire neural network.

\subsection{Data structures}
% \marginpar{Describe data structures you use in your project (lists, trees, heaps, , \ldots). Explain your choice. Add a figure.}
Program has used a lot of vectors and objects of custom classes. Vectors were used to store serial data and other structures. I implemented Neuron as a class as well to make calculations easier and to order them to the proper objects.

\subsection{Algorithms}
The program loads input, create an instance of Neural Network Class. While calling a constructor of the Neural Network, it creates Neurons of the specified amount in the constructor parameter. When the train method is called then Network is iterating throughout the input dataset and call the loss of every iteration(epoch) and on its basis adjust the weights and biases of every Neuron.

%%%%%%%%%%%%%%%%%%%%%%%%%%%%%%%%%%%%%%%%%%%%%%%%%%%%%%%%%%%%%%%%%%%%%%%%%
\section{External specification}
\label{sec:external}
This is a command line program.%
You can execute a program by using compiled .exe file or by using make in the e.g. Bash terminal.
The program requires input datasets specified in the main.cpp

Program call
\begin{verbatim}
program 
program -h
\end{verbatim}
prints a short manual. Program called with incorrect parameters prints an error message and prints help.

After execution you can find output file in the following direction:
\begin{verbatim}
/logs/output.csv
\end{verbatim}

You have to ensure yourself '/logs' directory exists, otherwise output file won't be created.


%%%%%%%%%%%%%%%%%%%%%%%%%%%%%%%%%%%%%%%%%%%%%%%%%%%%%%%%%%%%%%%%%%%%%%%%%
\section{Internal specification}\label{sec:internal}
The program is implemented with object-oriented, structural and functional paradigm. User interface is separated from program's logic.

 

\subsection{Program overview}
The \lstinline|main| function checks number of arguments passed to the program and creates a new Options object, which constructor check parameters of the program. If the verification is negative, an appropriate message is printed. In case of positive verification, data are read and saved into vector, which is passed, simultaneously with the options object, to the constructor of an \lstinline|GenerationsFactory| object. Then \lstinline|GenerationsFactory::makeGenerations| method performs genetic algorithm by creating first generation using objects of classes: \lstinline|Generation|, \lstinline|Individual|, and then it creates every another generation of individuals. Finally the program prints data into an output file.

\subsection{Description of types and functions}
Description of types and functions is moved to the appendix.

 

\section{Testing}
% \marginpar{Describe how you have tested your program.}
The program has been tested with various types of files. Incorrect files (with no numbers, numbers in incorrect format, strings with some invalid whitespaces, \ldots) are detected and an error message is printed. An empty input file does not cause failure – an empty output file is created. Maximal number value(\lstinline!double!) in an input file is approximately 1.8e+308. Maximal input file size handled by the program is \mbox{1.57$\,$GB}. Larger files result in a bad allocation error. The program has no memory leaks.
 


\section{Conclusions}

The program implements a solution to a knapsack problem using genetic algorithm. The most challenging task is creating a crossover system where a lot of data can be processed iteratively.

For some parameters the program elaborates incorrect results on some machines. This is caused by specification of the algorithm which does not includes an evolution coeeficient. Algorithm also base on a small dose of randomness, and it can be succesfull if generation/individuals number is insufficient.


\bibliographystyle{plain}
\bibliography{\jobname}

 
 
\cleardoublepage

\rule{0cm}{0cm}

\vfill

\begin{center}
\Huge\bfseries Appendix\\Description of types and functions\par
\end{center}

\vfill 

\rule{0cm}{0cm}

\end{document}
% Koniec wieńczy dzieło.
